% !TeX spellcheck = de_DE_frami
\documentclass[12pt]{article}
%\documentclass[12pt,a4paper,titlepage]{scrartcl}
%\usepackage[left=2cm,right=2.5cm,top=2.5cm,bottom=2cm]{geometry}
\usepackage[onehalfspacing]{setspace}
\usepackage[ngerman]{babel}
\usepackage[hidelinks]{hyperref}
\usepackage{verbatim} % for multiline comments
\usepackage{graphicx}
\graphicspath{ {./images/} }

\begin{document}

\begin{titlepage}	
	\title{\LARGE Lernzirkel Management System Manual}
	\date{\small \today}
	\author{\small Carl Heinrich Bellgardt}	
	\clearpage\maketitle
	\thispagestyle{empty}
\end{titlepage}

\setcounter{page}{2}
\tableofcontents
\pagebreak

\section{Installation}
	So funktioniert das Aufsetzen eines Lernzirkel-Servers.
\section{Allgemeine Benutzung}
\subsection{Schüler}
	\subsubsection{Hinzufügen eines neuen Schülers}
		Neuer Schüler.
	\subsubsection{Bearbeiten eines Schüler}
		So bearbeitet man.
	\subsubsection{Löschen eines Schülers}
		So löscht man.
	\subsubsection{Archivieren eines Schülers}
		Deshalb Archiviert man und so geht es.
\subsection{Mit Terminen ungehen}
	\subsubsection{Einen neuen Termin planen}
		So fügt man einen neuen einmaligen doer regelmäßigen Termin hinzu.
	\subsubsection{Den Status eines Termins ändern}
		Status.

\end{document}

